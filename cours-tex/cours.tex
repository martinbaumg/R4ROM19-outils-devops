\documentclass[12pt, a4paper]{article}
\usepackage[francais]{babel}
\usepackage{caption}
\usepackage{graphicx}
\usepackage[T1]{fontenc}
\usepackage{listings}
\usepackage{geometry}
\usepackage[colorlinks=true,linkcolor=black,anchorcolor=black,citecolor=black,filecolor=black,menucolor=black,runcolor=black,urlcolor=black]{hyperref}

% \usepackage{mathpazo} --> Police à utiliser lors de rapports plus sérieux

\usepackage{fancyhdr}
\pagestyle{fancy}
\lhead{}
\rhead{}
\chead{}
\rfoot{\thepage}
\lfoot{Martin Baumgaertner}
\cfoot{}

\renewcommand{\headrulewidth}{0.4pt}
\renewcommand{\footrulewidth}{0.4pt}

\begin{document}
\begin{titlepage}
	\newcommand{\HRule}{\rule{\linewidth}{0.5mm}} 
	\center 
	\textsc{\LARGE iut de colmar}\\[6.5cm] 
	\textsc{\Large R4ROM19}\\[0.5cm] 
	\textsc{\large Année 2022-23}\\[0.5cm]
	\HRule\\[0.75cm]
	{\huge\bfseries Outils DevOps}\\[0.4cm]
	\HRule\\[1.5cm]
	\textsc{\large martin baumgaertner}\\[6.5cm] 

	\vfill\vfill\vfill
	{\large\today} 
	\vfill
\end{titlepage}
\newpage
\tableofcontents
\newpage
\section{CM 1 - 9 mars 2023}
\subsection{Introduction}
Qu'est-ce que DevOps ? C'est un mélange entre une équipe de développement mais 
aussi un Opérateur, qui va s'occuper de la mise en production. C'est pour 
améliorer la qualité du code, la rapidité de mise en production et de livraison.\\ 

\textbf{Les objectifs d'un DevOps :}\\
\begin{itemize}
    \item Améliorer la qualité du produit
    \item Améliorer la rapidité de mise en production
    \item Améliorer la rapidité de livraison
    \item Réduire le coût grâce à l'automatisation
    \item Améliorer la sécurité
\end{itemize}

\subsection{Les outils de conténérisation}
\subsubsection{Docker}
Docker est un outil de conténérisation. Il permet de créer des conteneurs qui
sont des environnements isolés. Il permet de créer des environnements de
développement, de test et de production.\\

\textbf{Les avantages de Docker :}\\
\begin{itemize}
    \item Permet de créer des environnements de développement, de test et de production
    \item Permet de créer des environnements isolés
    \item Permet de créer des environnements légers
    \item Permet de créer des environnements répliquables
    \item Permet de créer des environnements scalables
\end{itemize}

\newpage
\subsubsection{Kubernetes}
Kubernetes est un outil de conténérisation. Il permet de créer des clusters
de conteneurs. Il permet de créer des environnements de développement, de test
et de production.\\

\textbf{Les avantages de Kubernetes :}\\
\begin{itemize}
    \item Permet de créer des environnements de développement, de test et de production
    \item Permet de créer des environnements isolés
    \item Permet de créer des environnements légers
    \item Permet de créer des environnements répliquables
    \item Permet de créer des environnements scalables
\end{itemize}

\subsection{YAML}
YAML est un langage de configuration. Il permet de décrire des objets. Il est
utilisé pour décrire des conteneurs, des clusters, des déploiements, des
services, des secrets, des configmaps, des volumes, des jobs, des cronjobs,
etc.\\

\newpage
\section{CM 2 - 24 mars 2023}
\subsection{Les besoins d'automatisation}
Un administrateur système ne programme pas, il va venir crée des scripts
pour tout automatiser un maximum. 

\end{document}